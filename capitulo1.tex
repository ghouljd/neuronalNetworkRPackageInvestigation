\chapter{Preliminares}

\section{Planteamiento y formulación del problema}

La producción de papa criolla (\textit{Solanum Phureja}), ha tenido muchos desafíos por parte de la industria, debido a los grandes intereses provenientes de los diferentes consumidores que la misma posee, siendo así que es uno de los productos agrícolas más consumidos y con mayor importancia en el mundo, después del arroz, el maíz y el trigo. Colombia es el mayor productor de papa criolla en Latinoamérica y por lo cual es un país que posee grandes exigencias industriales para su producción (Ligarreto \textit{et al}, 2003).\\

Los tubérculos de papa criolla tienen un tamaño aproximado entre dos y ocho centímetros (2-8cm), y se pueden clasificar en tres calibres según su diámetro transversal promedio,  de donde radican los intereses comerciales de la misma. Los tubérculos de entre dos y medio y cuatro centímetros son preferidos para encurtidos y pre-cocidos mientras los tubérculos promedio entre cuatro y seis y medio centímetros son preferidos para frituras en hojuelas o con más de cinco centímetros para frituras en tiras (CORPOICA, 2009).\\

El crecimiento vegetal es definido por Cabezas (2005) como "`El aumento irreversible del tamaño y peso seco de las plantas (altura, área foliar, diámetro, número de células y cantidad de protoplasma) o los cambios que ocurren en una planta o población de plantas a través del tiempo, fenómeno acompañado del aumento en la complejidad estructural metabólica del organismo (diferenciación celular, número de hojas), por procesos de división y alargamiento celular, incorporación de materia y energía del ambiente (fotosíntesis, absorción de agua y de iones) y metabolización subsiguiente, la cual se traduce en multiplicación y diferenciación celular. Este proceso está íntimamente relacionado con algunos factores internos como fotosíntesis, respiración, transpiración, condiciones de estrés, concentración enzimática, balance hormonal y expresión genética"'.\\

En la papa son dos los procesos fisiológicos asociados directamente al rendimiento de la misma, la fotosíntesis y la respiración. Y estos dos son procesos íntimamente asociados, ya que durante la fotosíntesis se producen carbohidratos que son consumidos durante la respiración, pero una gran cantidad de estos carbohidratos contribuyen al proceso de producción incrementando el tamaño de los tubérculos, el nivel y periodo de crecimiento de los tubérculos es una variable que responde directamente a la expresión de rendimiento expresada como producción por día, siendo así que cerca de un noventa por ciento (90\%) del peso acumulado de los tubérculos producidos por una planta es producto de la asimilación de dióxido de carbono, así como otros procesos similares (Beukema, 1979). Sin embargo, estos procesos se ven afectados por condiciones externas tales como la intensidad de luz, temperatura, longitud de los días, condiciones del suelo que son variables no controladas, o por condiciones como la cantidad de riego, fertilizantes aplicados para la mejora de los suelos que son variables controlables pero que requieren de costos adicionales de producción (Pineros, 2009). Es por ello que se sigue buscando, con el manejo de variables controladas,  el control sobre la cantidad total producida y el tamaño de los tubérculos,  sin producir costos adicionales, siendo la densidad de siembra una de estas variables, en la cual nos enfocaremos en esta investigación (Bernal, 2017).\\

Una red neuronal artificial (ANN) es un modelo que crea una relación entre una configuración de señales de entrada y una señal de salida usando un modelo derivado de nuestro entendimiento de como el cerebro responde a determinados estímulos de entradas sensoriales. Así como un cerebro usa ese conjunto de células interconectadas llamadas neuronas, así las ANN usan una red de neuronal artificiales o nodos para solucionar problemas aprendidos (Lantz, 2015).\\

En general, las ANN son aprendices versátiles que pueden ser aplicarse a casi cualquier tarea de aprendizaje en cualquier contexto, clasificación, predicción, reconocimiento de patrones e incluso pudiéramos hablar de tareas de supervisión de procesos. Las ANN, se aplican mejor a los problemas cuyos datos de entrada y salida son bien definidos o bastante simples, sin embargo el proceso que relaciona la entrada con una determinada salida es en extremo complejo, por eso se conocen como un método de caja negra (Lantz, 2015).\\

De una manera muy arcaica, las ANN han sido usadas desde hace cincuenta años para simular como el cerebro humano es capaz de resolver problemas. Primero surgieron las simples funciones lógicas AND y OR, que ayudaron a los científicos a entender como biológicamente el cerebro opera. Sin embargo, como el avance tecnológico y computacional ha sido incrementado y continua creciendo potencialmente durante los últimos años, las ANN ha incrementado su complejidad siendo usadas frecuentemente en múltiples problemas como:

\begin{enumerate}
    \item{Programas de reconocimiento de voz y escritura, como los que usan los correos de voz
de los servicios de transcripción y máquinas clasificadoras de correo postal.}
	\item{La automatización de dispositivos inteligentes como los controles ambientales de un edificio de oficinas o automóviles auto dirigidos y drones auto guiados.}
	\item{Modelos sofisticados de patrones climáticos y otros fenómenos científicos, sociales o económicos.}
	\item{La clasificación y reconocimiento de cualquier tipo de frutos y hojas de plantas, basado en sus atributos y características.}
\end{enumerate} 

Las redes neuronales probabilísticas (PNN) se derivan de las ANN y se diferencian en que estas hacen uso de redes de funciones de base radial. Son adoptadas en muchos casos por su fácil entrenamiento y velocidad de análisis. Las PNN tienen tres capas, una primera de entrada, una de base radial y otra de competición. Básicamente la capa de base radial evalúa las distancias entre los valores de un vector de entrada y estas distancias son escaladas no linealmente, para que la capa de competitividad encuentre la distancia más corta, a través de la cual realiza la clasificación.

R es un lenguaje de programación y entornos para gráficos y cálculos estadísticos, es una implementación diferente al lenguaje S (Chambers, 1998), y se desarrolla bajo un proyecto de \textit{software} bajo Licencia Pública General (\textit{General Public License}, GNU) de la \textit{Free Software Foundation} (\url{https://www.fsf.org/}) en forma de código fuente. Se compila y se ejecuta en una amplia variedad de plataformas entre ellas \textit{FreeBSD},\textit{Linux}, \textit{Windows} y\textit{MacOS}. (The R Project for Statistical Computing, 2018).\\

El paquete \textit{neuralnet} es un paquete del lenguaje R, que permite la aplicación, visualización e implementación de redes neuronales. El paquete permite configuraciones flexibles a través de la elección personalizada del error y la función de activación. Además, el cálculo de pesos generalizados esta implementado (CRAN-R Project, 2016).\\

Esta propuesta permitirá ajustar y entrenar una red neuronal artificial probabilística que permita la clasificación de densidades de siembra de papa criolla (\textit{Solanum Phureja}) teniendo como entrada las siguientes variables: 

\begin{enumerate}
    \item{\textbf{Peso total del cultivo.} Variable cuantitativa expresada en gramos (gr), que indica el peso de todos los tubérculos recogidos en la siembra.}
	\item{\textbf{Total de tubérculos recogidos.} Variable cuantitativa que indica el número de tubérculos recogidos en la siembra.}
	\item{\textbf{Diámetro ponderado de todos los tubérculos.} Variable cuantitativa expresada en centímetros (cm.), que indica el cálculo del diámetro ponderado (por los tubérculos ser ovoides) de cada tubérculo recogido en la siembra.}
\end{enumerate}

La densidad de siembra es la distancia entre plantas y leras o sencillamente cuantas plantas son sembradas por metro cuadrado de siembra.\\

El objetivo es comparar los resultados de clasificación alimentando la red con datos observados y concluir hasta qué punto la densidad de siembra afecta las variables antes mencionadas. Otra de las pruebas que se plantea usar en esta investigación es realizar los análisis eliminando variables de entrada atisbando cuales variables permiten y generan una mejor clasificación.\\

\section{Objetivos}

\subsection{Objetivo General}

Desarrollar un paquete en lenguaje R que permita la clasificación de tubérculos de papa criolla (\textit{Solanum Phureja}) para diferentes densidades de siembra empleando redes neuronales probabilisticas.

\subsection{Objetivos Específicos}
 
\begin{itemize}
\item{Estudiar la parametrización de la densidad de siembra como parámetro de entrada al clasificador}
\item{Diseñar los algoritmos para las funciones que permitan el entrenamiento de una red neuronal probabilística que permita la clasificación de tubérculos de papa criolla (\textit{Solanum Phureja}) para diferentes densidades de siembra.}
\item{Implementar las funciones  para la clasificación de tubérculos de papa criolla para diferentes densidades de siembra, basadas en los algoritmos desarrollados.}
\item{Realizar la pruebas del paquete desarrollado bajo diferentes escenarios.}
\end{itemize}

\section{Justificación e Importancia}

Colombia es el mayor productor, consumidor y exportador de papas diploides en el mundo; tiene una ventaja competitiva notable debido a ser centro de diversidad y poseer gran aceptación por los consumidores debido a las características del tubérculo. Adicionalmente, en este país se ha desarrollado una amplia tradición como cultivo tecnificado, con potencial de industrialización y exportación (Rodríguez \textit{et al}, 2009). Además de contar con uno de los recursos genéticos que ofrece las mayores oportunidades de exportación como alimento procesado exclusivo y sin competencia. Desafortunadamente, muchas de las estrategias de manejo del cultivo de la papa, han sido adaptadas a papa criolla, creando un sistema productivo poco eficiente (Piñeros, 2009). Los altos precios de los insumos agrícolas y el mal manejo de los cultivos agronómicamente provoca  baja productividad y amenaza la competitividad del sistema de producción, por lo que es importante identificar factores limitantes del rendimiento y desarrollar practicas innovadoras para el cultivo, tales como una gestión nutricional integrada y equilibrada, que es una de las practicas más eficientes para garantizar a la planta la oportunidad de expresar su potencial genético que eventualmente se reflejara en una mejor calidad y rendimiento (Lopez \textit{et al}, 2014).\\
 
Varias investigaciones revelan que el tamaño promedio de los tubérculos, su variabilidad y su conteo, definen una única distribución, presentando mayor variabilidad de rangos de tamaño en la etapa final de llenado, con aumento del rendimiento de los mayores tamaños por cuenta de los tamaños inferiores, lo que sugiere que un tubérculo que pasa de un calibre inferior al siguiente, no es sustituido por otro de calibre anterior, además, parece existir una correlación negativa entre la variabilidad relativa del tamaño del tubérculo y el número de tubérculos por unidad de área (Struik, 1991).\\

Es importante reconocer, que debe tenerse cuidado con la interpretación de esta correlación, pues hasta cierto punto, la naturaleza de los datos asociados al número de tubérculos por planta pudiera ser similar a la de los datos composicionales, es decir, si un componente aumenta, el otro esta forzado a disminuir, lo que genera correlaciones sin conexión lógica en datos verdaderamente composicionales, sin embargo, el número de tubérculos por calibre no suma una constante, pudiera ocurrir que su máximo posea una distribución específica, entonces una mayor cantidad de tubérculos de un calibre podría provocar un número menor de tubérculos en otro calibre especifico, lo que en esencia puede generar correlaciones en las que debe tenerse mucho cuidado al momento de su interpretación.\\

 La importancia del cultivo obliga a muchos investigadores a realizar diferentes estudios para mejorarlo según el uso que vaya a darse a los tubérculos, generando nuevas variedades resistentes a plagas y enfermedades y de fácil adaptación a diferentes climas, procurando hacer uso del espacio de siembra de forma óptima, por lo que varios estudios evidencian el estudio de la densidad de siembra para evaluar sobre todo el rendimiento. En 2017, Bernal evaluó en lugar del rendimiento, un indicador que se asocia directamente como lo es el calibre de los tubérculos, los cuales en la práctica se manipulan en cuatro categorías de diámetro promedio (hasta 2 cm, de 2 a 4 cm, de 4 a 6 cm y más de 6 cm). La naturaleza de esta variable imposibilita usualmente la comparación de la respuesta (conteos de tubérculos) para cada densidad de siembra (definidas como 30cm*100cm, 40cm*100cm y 50cm*100cm) mediante análisis de varianza, pues en el caso de conteos, otras distribuciones como la Poisson y la Binomial negativa se adaptan mucho mejor al tipo de dato generado. Los modelos clásicos de regresión Poisson y binomial y negativa, en sus modalidades usuales o en la opción inflada por ceros (Cameron, 1998) en datos de conteo pertenecen a la familia de modelos lineales generalizados (Zeileis, 2008) y los desarrollos recientes permiten generar varias estadísticas que permiten su comparación con sus contrapartes sin ceros en exceso, lo que resulta útil en la elección del mejor modelo para relacionar predictores y respuesta en conteos.\\

La propuesta consiste en clasificar los datos a través de una red neuronal que no amerita de suposiciones estadísticas o matemáticas, que sólo aprendiendo de lo observado es capaz de realizar una clasificación, a partir de la información existente de la relación latente entre de densidades de siembra y el tamaño del tubérculo, que como herramienta descriptiva pueda ayudar a la planificación de los procesos de siembra.\\



\section{Alcance y Limitaciones}

La red neuronal artificial probabilística a implementar tiene como propósito la clasificación de la densidad de siembra en papa criolla(\textit{Solanum Phureja}), siendo útil para incrementar las posibilidades de obtener un determinado y deseado calibre de tubérculos conociendo la densidad a la que se debe sembrar.  Las pruebas se realizarán bajo solo un conjunto de datos reales que fueron tomados bajo las medidas de cuatro calibres de tubérculos y tres densidades de siembra.\\

Los datos disponibles del cultivo de papa se tienen en base a tres densidades de siembra, cantidad de plantas sembradas, cantidad de tubérculos recogidos, peso total del cultivo, diámetro ponderado de cada tubérculo y calibre de los tubérculos y forman parte de la investigación de Maestría del estudiante de la Universidad Nacional de Colombia, Nelson Bernal (2017).\\
\\

