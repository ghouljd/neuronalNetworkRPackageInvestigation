

\chapter*{Introducci\'on}
\pagenumbering{arabic} % para empezar la numeración con números

En los cultivos de papa criolla son muchas las variables que influyen en la cosecha de tuberculos, tales como la temperatura, altura del terreno, clima, tipo y composicion del suelo, densidad de siembra, entre otros y los encargados y personas interesadas en este proceso siempre han buscado manipularlas para no solo mejorar la calidad del producto, si no para obtenerlo con ciertas caracteristicas que mas convengan para su fin como la piel, contextura, pero muy sobre todo el tamano de los tuberculos ya que existen intereses industriales diferentes en esta ultima caracteristica.\\

Investigaciones anteriores como al realizada por Bernal (2017), concluyeron que la densidad de siembra es una variable con una afectacion considerable en el tamano de los tuberculos producidos por las plantas en su ciclo de cosecha, sin embargo la manipulacion de esta variable en dicha investigacion, fue de tipo lineal sin tener en cuenta, ciertas condiciones espaciales que posee como propiedades esta variable, por la misma ser distancias entre calles y surcos, y distancia entre plantas a la que se siembra o expresado de otra manera cuantas plantas son sembradas por metro cuadrado en un terreno.\\

La regresion espacial es una solucion que se plantea usar en el siguiente documento para el modelado de los datos recogidos y observados de la cosecha de un cultivo de tuberculos de papa criolla realizado en el Centro Agropecuario Marengo de la Universidad Nacional de Colombia, en el departamento de Cundinamarca en Colombia (74°12'58.51''W;4°40'52.92''N), el cual tiene una altitud de 2516 msnm y una temperatura media de 14 °C.\\

La densidad de siembra es una variable cualitativa que aunque se trata de distancias se definen en densidades predeterminadas como d1 = 50x30cm, donde d1 es una densidad definida igual a un espacio entre surcos de 50cm y distancia entre plantas de 30cm. Haciendo que el analisis de regresion espacial deba hacerse por separado para los datos de cada densidad tomados, lo que nos presenta ante la dificultad de no poder asegurar una relacion entre los resultados porque aunque los datos tengan el mismo comportamiento al ser modelados podrian no tener una normalizacion bivariante.\\

Las redes neuronales clasifican basado en sus neuronas entrenadas a traves de datos conocidos con anterioridad por lo que no es de importancia el hecho de que no exista una relacion entre estas variables ya que no requiere de supuestos estadisticos a tener en cuenta para su proceso de clasificacion.\\

El objetivo de esta investigacion es construir un paquete en Lenguaje R que permita a traves de una entrada, propuesta como una matriz cargada con los datos observados de cultivo, clasificar los tuberculos de papa criolla cosechados en densidades de siembra segun el peso total del cultivo, peso total de tuberculos y diametro de cada tuberculo a traves de una red neuronal probabilistica entrenada con datos del cultivo anteriormente recogidos, ademas de comparar los resultados provenientes de una regresion espacial versus datos observados, para concluir cuan efectivo puede ser aplicar una regresion espacial para estos casos.\\