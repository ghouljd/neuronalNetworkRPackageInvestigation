

\chapter*{Introducci\'on}
\pagenumbering{arabic} % para empezar la numeración con números

En los cultivos de papa criolla son muchas las variables que influyen en la cosecha de tub\'erculos, tales como la temperatura, altura del terreno, clima, tipo y composici\'on del suelo, densidad de siembra, entre otros y los encargados y personas interesadas en este proceso siempre han buscado manipularlas para no solo mejorar la calidad del producto, si no para obtenerlo con ciertas caracter\'isticas que mas convengan para su fin como la piel, contextura, pero muy sobre todo el tama\~no de los tub\'erculos ya que existen intereses industriales diferentes en esta \'ultima caracter\'istica.\\

Investigaciones anteriores concluyeron que la densidad de siembra es una variable con una afectaci\'on considerable en el tama\~no de los tub\'erculos producidos por las plantas en su ciclo de cosecha, sin embargo la manipulaci\'on de esta variable en dichas investigaci\'ones, fue de tipo lineal sin tener en cuenta, ciertas condiciones espaciales que posee como propiedades esta variable, por la misma ser distancias entre calles y surcos, y distancia entre plantas a la que se siembra o expresado de otra manera cuantas plantas son sembradas por metro cuadrado en un terreno.\\

El uso de redes neuronales probabil\'isticas es una soluci\'on que se plantea usar en el siguiente documento para el modelado de los datos recogidos y observados de la cosecha de un cultivo de tub\'erculos de papa criolla realizado en el Centro Agropecuario Marengo de la Universidad Nacional de Colombia, en el departamento de Cundinamarca en Colombia (74°12'58.51''W;4°40'52.92''N), el cual tiene una altitud de 2516 msnm y una temperatura media de 14 °C.\\

La densidad de siembra es una variable cualitativa que aunque se trata de distancias se definen en densidades predeterminadas como d1 = 50x30cm, donde d1 es una densidad definida igual a un espacio entre surcos de 50cm y distancia entre plantas de 30cm.\\ 

Las redes neuronales clasifican basado en sus neuronas entrenadas a trav\'es de datos conocidos con anterioridad por lo que no es de importancia el hecho de que no exista una relacion entre estas variables ya que no requiere de supuestos estad\'isticos a tener en cuenta para su proceso de clasificaci\'on. El objetivo de esta investigaci\'on es construir un paquete en Lenguaje R que permita a trav\'es de una entrada, propuesta como una matriz cargada con los datos observados de cultivo, clasificar los tub\'erculos de papa criolla cosechados en densidades de siembra seg\'un el peso total del cultivo, peso total de tub\'erculos y diam\'etro de cada tub\'erculo a trav\'es de una red neuronal probabil\'istica entrenada con datos del cultivo anteriormente recogidos.\\