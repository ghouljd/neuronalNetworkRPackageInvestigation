\chapter{Fundamentos Metodol\'ogicos}

	A continuaci\'on se plantea la estructura a seguir por el presente trabajo, detallando el enfoque, tipo, nivel y dise\~no de la investigaci\'on y la metodolog\'ia a implementar entre otros.
	
\section{Enfoque de la investigaci\'on}
	
	La presente investigaci\'on se desarrollar\'a siguiendo un enfoque cuantitativo, puesto que, como lo indican Pallela y  Martins (2012) , “la investigaci\'on cuantitativa requiere el uso de instrumentos de medici\'on y comparaci\'on, que proporcionan datos cuyo estudio necesita la aplicaci\'on de m\'odelos matem\'aticos y estad\'isticos, el conocimiento est\'a basado en hechos”.  Los datos a usar en el desarrollo de este trabajo investigativo fueron recolectados directamente de un cultivo y forman parte de la investigacion de Maestria del estudiante de la Universidad Nacional de Colombia, Nelson Bernal.\\
	
\section{Tipo o nivel de investigaci\'on}
	
	Este proyecto plantea un tipo de investigaci\'on de campo, seg\'un como lo indican Pallela y  Martins (2012),  la investigaci\'on de campo “consiste en la recolecci\'on directamente de la realidad donde ocurren los hechos, sin manipular o controlar variables” ya que permite indagar los efectos de la interrelaci\'on entre los diferentes tipos de variable en lugar de los hechos.\\

	En este punto se debe determinarla profundidad que abarca esta investigaci\'on, teniendo en cuenta que de acuerdo con  el nivel de la investigaci\'on es definido como “grado de profundidad con que se aborda un fen\'omeno u objeto de estudio” (Arias, 2012).\\

	En este sentido, se tiene que dadas las caracter\'isticas del proyecto, se asocia con un nivel descriptivo, tal como lo indican Pallela y  Martins (2012),  “hace \'enfasis sobre conclusiones dominantes o sobre como una persona, grupo o cosa se conduce o funciona en el presente” esto debido a que se medir\'an los datos extra\'idos sin alterarlos para ser mostrados en el sistema.\\

	Cuando se habla de un nivel descriptivo junto con una investigaci\'on de tipo de campo, en ella no se formulan hip\'otesis y las variables se enuncian en los objetivos de la investigaci\'on que se desarrollar\'a.
	
\section{Dise\~no de la investigaci\'on}
	
Seg\'un Arias(2012), el dise\~no de la investigaci\'on es “la estrategia general que adopta el investigador para responder al problema planteado” (p.21) por lo que es vital establecer una correcta secuencia de pasos para elaborar el prototipo de software que dar\'a soluci\'on a la problem\'atica principal de la investigaci\'on.\\

Con este enfoque, se tiene que este trabajo seguir\'a un dise\~no no experimental, enfocado en el uso de informaci\'on existente, de acuerdo con lo dicho por Pallela y  Martins (2012) al definir el dise\~no no experimental como:\\

Es el que se realiza sin manipular en forma deliberada ninguna variable. El investigador no sustituye intencionalmente las variables independientes. Se observan los hechos tal y como se presentan en su contexto real y en un tiempo determinado o no, para luego analizarlos. Por lo tanto, este dise\~no no se construye una situaci\'on espec\'ifica sino que se observan las que existen. Las variables independientes ya han ocurrido y no pueden ser manipuladas, lo que impide influir sobre ellas para modificarlas. (p.81)\\

Esto indica que no hay manipulaci\'on de variables. Esta investigaci\'on presenta una modalidad de proyecto especial que, como lo indican Pallela y  Martins (2012), los proyectos especiales “destinados a la creaci\'on de productos que puedan solucionar deficiencias evidenciadas, se caracterizan por su valor innovador y aporte significativo” (p.92), ya que se crear\'a un \emph{software} aplicable al \'area de estudio.\\

\section{Metodolog\'ia}

Los pasos a seguir en el desarrollo de esta investigacion seran descritos a continuacion siendo basados en los antecedentes y estudios realizados.\\

\noindent
\textbf{Explorar y manipular la data.}\\

Los datos tomados de la investigacion de Bernal, son datos reales observados de un cultivo de tuberculos de papa criolla que no han sido tratados estadisticamente o modelados, es por eso que en esta etapa se disenara un algoritmo para la carga de los mismos en estructuras de datos en R, para luego ser modelados y analizados espacialmente por medio de regresion espacial y probando de igual forma su normalizacion bivariante.\\

\noindent
\textbf{Disenar y entrenar la red neuronal probabilistica.}\\

En esta estapa se definiran las variables de entrada y se declararan las neuronas patrones y las clases de salida que permitiran la clasificacion, para proceder con el entrenamiento de la red que se hara a traves del metodo de jackknifing para determinar el parametro de escalamiento correcto que permita la mejor clasificacion.\\

\noindent
\textbf{Implementar y realizar pruebas de clasificacion.}\\

El objetivo de esta etapa es definir los casos de prueba que seran usados para buscar la mejor clasificacion de densidad de siembra de tuberculos de papa criolla y realizar las pruebas con los diferentes escenarios a preparar.\\

\noindent
\textbf{Comparacion de resultados.}\\

En este paso se procedera a la construccion de las curvas ROC asociadas a los casos de prueba planteados para asi obtener las mejores clasificaciones, permitiendo concluir las mejores relaciones entre variables, el mejor manejo de los datos y si los mismos deben ser modelados o es suficientemente buena una clasificacion con datos observados a pesar de la dependencia espacial existente.\\

\noindent
\textbf{Graficar y documentar resultados.}\\

Finalmente se procedera a la graficacion por medio de R de los resultados, que sustenten las conclusiones realizadas y a relizar la documentacion que permita el uso de esta investigacion en futuras aplicaciones de campo.\\  

\section{Aspectos administrativos}

\vspace{1 cm}
La realización de la investigación será planificada según lo establecido en el siguente diagrama:\\

\begin{figure}[!ht]
\begin{center}

\begin{ganttchart}[y unit title=0.4cm,
y unit chart=0.5cm,
vgrid,hgrid,
title height=1,
bar/.style={draw,fill=cyan},
bar incomplete/.append style={fill=yellow!50},
bar height=0.7]{1}{16}
 \gantttitle{Semanas}{16}\\
 \gantttitle{Julio}{2}
 \gantttitle{Agosto}{4} 
 \gantttitle{Septiembre}{4}
 \gantttitle{Octubre}{5}\\
 \gantttitle{Noviembre}{1}\\
 \gantttitlelist{1,...,2}{1}
 \gantttitlelist{1,...,4}{1}
 \gantttitlelist{1,...,4}{1} 
 \gantttitlelist{1,...,5}{1} 
 \gantttitlelist{1,...,1}{1}\\
 \ganttbar{\tiny{Explorar y manipular la data}}{1}{2} \\
 \ganttbar{\tiny{Disenar y entrenar la red neuronal probabilistica}}{3}{7} \\
 \ganttbar{\tiny{Implementar y realizar pruebas de clasificacion.}}{8}{11} \\
 \ganttbar{\tiny{Comparacion de resultados}}{12}{14} \\
 \ganttbar{\tiny{Graficar y documentar resultados}}{15}{16} \\
 \ganttbar{\tiny{Realización del informe del proyecto especial de grado}}{2}{16} \\
 \ganttlink{elem0}{elem1}
 \ganttlink{elem1}{elem2}
 \ganttlink{elem2}{elem3}
 \ganttlink{elem3}{elem4}
\end{ganttchart}

\end{center}
\caption{Diagrama de Gantt con la planficación del proyecto especial de grado}
\end{figure}


	