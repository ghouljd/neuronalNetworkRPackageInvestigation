\chapter{Fundamentos te\'oricos}

\section{Antecedentes}

La clasificacion por medio de redes neuronales ha sido un hito ya marcado en el campo agronomico, muchos estudios se han realizado con el objetivo de analizar ciertos comportamientos de plantas y los beneficios que se puedan sacar de ellas. En 2011, el grupo de investigacion de sistemas de procesamiento y control de senales de la Universidad Nacional Tenaga de Malasia desarrollo un sistema de inteligencia con un enfoque novedose para la clasificacion de frutas usando tecnicas de procesamiento de imagenes digitales y redes neuronales artificiales, con el objetivo de desarrollar un metodo de clasificacion rapido con una meta del 100\% de eficiencia. El estudio se realizo con cinco frutas, manzanas, platanos, zanahorias, mangos y naranjas, extrayendo de ellas siete caracteristicas en funcion de la forma y el color. La captura de las imagenes se realizo con una camara digital convencional y las manipulaciones a los datos y construccion de la red con el software MATLAB. Los resultados obtenidos durante esta investigacion fueron de gran avance en el campo de reconocimiento de patrones en imagenes.\\

En 2011, Mayabiro E presento en la Universidad Nacional Experimental del Tachira, un prototipo sobre el entorno MATLAB para el calculo de la tasa de germinacion de plantulas de pimenton previamente segmentadas. El entorno desarrollado permitia establecer una clasificacion de las plantulas, hojas u objetos de la misma por medio de redes neuronales. Se realizo el entrenamiento de multiples redes neuronales multicapas con algoritmos de retropropagacion, donde aunque variaban las capas intermedias de las redes y sus funciones de transferencia fueron entrenadas con los mismos  datos de entrada, validandolas con una base de datos de pruebas para seleccionar al final una con salidas similares a las deseadas.\\

Otro estudio realizado en 2013 por Stephen Gang Wu consistia en el estudio teorico de tecnicas de procesamiento de imagenes y datos para el reconocimiento automatico de hojas para la clasificacion de plantas. Doce caracteristicas de las plantas fueron extraidos y distribuidas en cinco variables principales que constituian el vector de entrada de una red neuronal artificial probalistica, que habis sido entrenada con 1800 hojas para clasificar 32 tipos de plantas con una precision superior al 90\%, el autor aseguraba que su metodologia de implementacion de la PNN era facil y rapida en comparacion de otras investigaciones similares.\\

En 2017, Bernal N realizo un estudio de campo con el cultivo de papa criolla para evaluar la influencia de la densidad de siembra asociada a distancias entre plantas de 30,40 y 50 cm y distancias entre surcos de 100 cm sobre el conteo de tuberculos de calibres inferiores a 2 cm, entre 2 y 4 cm, entre 4 y 6 cm, y de mas de 6 cm de diametro ponderado y sobre el peso fresco en gramos de los tuberculos. Los tuberculos cosechados se clasificaron y contaron mediante tamizado y se pesaron en su totalidad sin discriminar por calibre. Los modelos estadisticos empleados para modelar el comportamiento de la cosecha, evidenciaron el efecto significativo de la densidad de siembra sobre el conteo de tuberculos y calibre y se observo una razon aproximada de 40:40:20:1 desde el calibre menor al mayor. El efecto de la competicion, en todos los modelos probados resulto significativo, aumentando en la mayoria de los casos a medida que disminuia la distancia entre plantas, tanto en el patron de vecindad intrahileras como en el caso de inter e intrahileras.

\section{Bases Te\'oricas}

\subsection{Lenguaje R.}

R es un conjunto  integrado de \textit{software} de código abierto para el almacenamiento, manipulación, cálculo y visualización de datos para computación y gráficación estadística, puede ser compilado y ejecutado en  en Windows, Mac OS X y otras  plataformas UNIX (como Linux), se distribuye usualmente en formato binario (\url{https://www.r-project.org/about.html}, 2018). El proyecto de \emph{software} R fue iniciado por Robert Gentleman y Ross Ihaka. El lenguaje fue influenciado por  lenguaje S desarrollado originalmente en Bell Laboratories por John Chambers y sus colegas. Desde entonces ha evolucionado  para el cálculo estadístico asociado a diversas disciplinas para contextos académicos y comerciales. En R, la unidad fundamental de código compartible es el paquete, el cual agrupa código, datos, documentación y pruebas, y resulta simple de compartir con otros. Para enero del 2015 ya habían más de 6.000 paquetes disponibles en la Red Integral de Archivos de R, conocido comunmente por su acrónimo CRAN, el cual es el repositorio de paquetes . Esta gran variedad de paquetes es una de las razones por las cuales R es tan exitoso, pues es probable que algún investigador o académico ya haya resuelto un problema en su propio campo usando esta herramienta, por lo que otros usuarios simplemente podrán recurrir a ella para su uso directo o para llamarla en un nuevo código (Wickham,2015). \\


\subsection{RStudio.}

RStudio es un ambiente de desarrollo integrado (\textit{Integrated Development Environment}, IDE) que ofrece herramientas de desarrollo vía consola, editor de sintaxis que apoya la ejecución de código, así como herramientas para el trazado, la depuración y la gestión del espacio de trabajo.  RStudio está disponible para Windows, Mac y Linux o para navegadores conectados a RStudio Server o RStudio Server Pro (Debian / Ubuntu, RedHat / CentOS, y SUSE Linux) (\url{https://www.rstudio.com/about/}, 2018)
 

\subsection{Glosario}

\paragraph{R}: 


