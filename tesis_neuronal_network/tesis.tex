\documentclass[oneside,12pt,numbers,spanish]{ezthesis}
%% # Opciones disponibles para el documento #
%%
%% Las opciones con un (*) son las opciones predeterminadas.
%%
%% Modo de compilar:
%%   draft            - borrador con marcas de fecha y sin im'agenes
%%   draftmarks       - borrador con marcas de fecha y con im'agenes
%%   final (*)        - version final de la tesis
%%
%% Tama'no de papel:
%%   letterpaper (*)  - tama'no carta (Am'erica)
 %%   a4paper          - tama'no A4    (Europa)
%%
%% Formato de impresi'on:
%%   oneside          - hojas impresas por un solo lado
%%   twoside (*)      - hijas impresas por ambos lados
%%
%% Tama'no de letra:
%%   10pt, 11pt, o 12pt (*)
%%
%% Espaciado entre renglones:
%%   singlespace      - espacio sencillo
%%   onehalfspace (*) - espacio de 1.5
%%   doublespace      - a doble espacio
%%
%% Formato de las referencias bibliogr'aficas:
%%   numbers          - numeradas, p.e. [1]
%%   authoryear (*)   - por autor y a'no, p.e. (Newton, 1997)
%%
%% Opciones adicionales:
%%   spanish         - tesis escrita en espa'nol
%%
%% Desactivar opciones especiales:
%%   nobibtoc   - no incluir la bibiolgraf'ia en el 'Indice general
%%   nofancyhdr - no incluir "fancyhdr" para producir los encabezados
%%   nocolors   - no incluir "xcolor" para producir ligas con colores
%%   nographicx - no incluir "graphicx" para insertar gr'aficos
%%   nonatbib   - no incluir "natbib" para administrar la bibliograf'ia

%% Paquetes adicionales requeridos se pueden agregar tambi'en aqu'i.
%% Por ejemplo:
\usepackage[spanish]{babel}
\usepackage{cite}
\usepackage{natbib} 
\usepackage{subfig}
\usepackage[utf8x]{inputenc}
\usepackage{multicol}
\usepackage{multirow, array}
\usepackage{graphicx} 
\usepackage{float}  
\usepackage{amsmath}
\usepackage{amssymb}
\usepackage{pgfgantt}


%% # Datos del documento #
%%\documentclass[12pt]{•}%% Nota que los acentos se deben escribir: \'a, \'e, \'i, etc.
%% La letra n con tilde es: \~n.

\author{Jesús David Escalante Rodr\'iguez}
\title{Paquete en lenguaje R para la clasificaci\'on de tub\'erculos de papa criolla(\textit{Solanum phureja}) para diferentes densidades de siembra empleando redes neuronales probabil\'isticas.}
\supervisor{ Rossana Timaure}
\institution{Universidad Nacional Experimental del T\'achira}
\faculty{Vicerectorado Acad\'emico}
\department{Departamento de Ingeniería en Informática}

\pretolerance = 2000
\tolerance = 3000
\righthyphenmin = 2000
\lefthyphenmin = 2000
%% # M'argenes del documento #
%% 
%% Quitar el comentario en la siguiente linea para austar los m'argenes del
%% documento. Leer la documentaci'on de "geometry" para m'as informaci'on.

\geometry{left=2cm,right=2cm,top=3cm,bottom=3cm}

%% El siguiente comando agrega ligas activas en el documento para las
%% referencias cruzadas y citas bibliogr'aficas. Tiene que ser *la 'ultima*
%% instrucci'on antes de \begin{document}.
\hyperlinking
\begin{document}

%% En esta secci'on se describe la estructura del documento de la tesis.
%% Consulta los reglamentos de tu universidad para determinar el orden
%% y la cantidad de secciones que debes de incluir.

%% # Portada de la tesis #
%% Mirar el archivo "titlepage.tex" para los detalles.
%% ## Construye tu propia portada ##
%% 
%% Una portada se conforma por una secuencia de "Blocks" que incluyen
%% piezas individuales de informaci'on. Un "Block" puede incluir, por
%% ejemplo, el t'itulo del documento, una im'agen (logotipo de la universidad),
%% el nombre del autor, nombre del supervisor, u cualquier otra pieza de
%% informaci'on.
%%
%% Cada "Block" aparece centrado horizontalmente en la p'agina y,
%% verticalmente, todos los "Blocks" se distruyen de manera uniforme 
%% a lo largo de p'agina.
%%
%% Nota tambi'en que, dentro de un mismo "Block" se pueden cortar
%% lineas usando el comando \\
%%
%% El tama'no del texto dentro de un "Block" se puede modificar usando uno de
%% los comandos:
%%   \small      \LARGE
%%   \large      \huge
%%   \Large      \Huge
%%
%% Y el tipo de letra se puede modificar usando:
%%   \bfseries - negritas
%%   \itshape  - it'alicas
%%   \scshape  - small caps
%%   \slshape  - slanted
%%   \sffamily - sans serif
%%
%% Para producir plantillas generales, la informaci'on que ha sido inclu'ida
%% en el archivo principal "tesis.tex" se puede accesar aqu'i usando:
%%   \insertauthor
%%   \inserttitle
%%   \insertsupervisor
%%   \insertinstitution
%%   \insertdegree
%%   \insertfaculty
%%   \insertdepartment
%%   \insertsubmitdate

\begin{titlepage}
  \TitleBlock{\includegraphics[scale=0.2]{unet.jpg} }  
  \TitleBlock{\small\insertinstitution}
  \TitleBlock[\small\bigskip]{\insertfaculty}
  \TitleBlock[\small\bigskip]{Decanato de Docencia}
  \TitleBlock[\small\bigskip]{\insertdepartment}
  \TitleBlock[\small\bigskip]{Trabajo de aplicaci\'on profesional}
  \TitleBlock[\small\bigskip]{Proyecto especial de grado}
    
 \TitleBlock{\scshape\inserttitle}
 
 \begin{flushright}
  \TitleBlock{ \small Autor(es):  \insertauthor} 
  \TitleBlock{ \small C.I.: 21.220.841}
  \TitleBlock{ \small jesusd.escalante@unet.edu.ve}
  \TitleBlock{\small Tutor(es):  \insertsupervisor}
  \TitleBlock{ \small rttg@unet.edu.ve}
 \end{flushright}
  \TitleBlock{\small San Crist\'obal, Julio de 2018.\insertsubmitdate}
  
\end{titlepage}

%% Nota 1:
%% Se puede agregar un escudo o logotipo en un "Block" como:
%%   \TitleBlock{\includegraphics[height=4cm]{escudo_uni}}
%% y teniendo un archivo "escudo_uni.pdf", "escudo_uni.png" o "escudo_uni.jpg"
%% en alg'un lugar donde LaTeX lo pueda encontrar.

%% Nota 2:
%% Normalmente, el espacio entre "Blocks" se extiende de modo que el
%% contenido se reparte uniformemente sobre toda la p'agina. Este
%% comportamiento se puede modificar para mantener fijo, por ejemplo, el
%% espacio entre un par de "Blocks". Escribiendo:
%%   \TitleBlock{Bloque 1}
%%   \TitleBlock[\bigskip]{Bloque2}
%% se deja un espacio "grande" y de tama~no fijo entre el bloque 1 y 2.
%% Adem'as de \bigskip est'an tambi'en \smallskip y \medskip. Si necesitas
%% aun m'as control puedes usar tambi'en, por ejemplo, \vspace*{2cm}.




% \tableofcontents
% \listoffigures


\chapter*{Introducci\'on}
\pagenumbering{arabic} % para empezar la numeración con números

En los cultivos de papa criolla (\textit{Solanum phureja}) son muchas las variables que influyen en la cosecha de tub\'erculos, tales como la temperatura, altura del terreno, clima, tipo y composici\'on del suelo, densidad de siembra, entre otros. Los encargados y personas interesadas en este proceso siempre han buscado manipularlas,  no sólo para mejorar la calidad del producto, si no para obtenerlo caracter\'isticas que sea convenientes, como el tama\~no de los tub\'erculos ya que existen intereses industriales diferentes para su comercializaci\'on.\\

Investigaciones anteriores concluyeron que la densidad de siembra es una variable con una afectaci\'on considerable en el tama\~no de los tub\'erculos producidos por las plantas en su ciclo de cosecha, sin embargo la manipulación de esta variable generalmente se maneja de forma lineal sin tener en cuenta, ciertas condiciones espaciales que posee como propiedades esta variable, por  ser distancias entre calles y surcos, y distancia entre plantas a la que se siembra, o expresado de otra manera cuantas plantas son sembradas por metro cuadrado en un terreno.\\

La densidad de siembra es una variable cualitativa que aunque se trata de distancias se definen en densidades predeterminadas como d$_1$ = 50x30cm, donde d$_1$ es una densidad definida igual a un espacio entre surcos de 50cm y distancia entre plantas de 30cm.\\



Técnicas.\\

 
Las redes neuronales clasifican basado en sus neuronas entrenadas a través de datos conocidos con anterioridad, por lo que no es de importancia el hecho de que no exista una relación entre estas variables, ya que no requiere de supuestos estad\'isticos a tener en cuenta para su proceso de clasificación.\\


El uso de redes neuronales probabil\'isticas es una soluci\'on que se plantea usar en la siguiente propuesta para el modelado de los datos recogidos y observados de la cosecha de un cultivo de tubérculos de papa criolla (\textit{S. phureja}) realizado en el Centro Agropecuario Marengo de la Universidad Nacional de Colombia, en el departamento de Cundinamarca en Colombia (74°12'58.51''W;4°40'52.92''N), el cual tiene una altitud de 2516 msnm y una temperatura media de 14 °C. Implementado como un paquete en Lenguaje R que permita a través de una entrada, propuesta como una matriz cargada con los datos observados de cultivo, clasificar los tub\'erculos de papa criolla cosechados en densidades de siembra seg\'un el peso total del cultivo, peso total de tub\'erculos y diam\'etro de cada tub\'erculo.\\

Este anteproyecto de trabajo de grado está dividido en los siguientes capítulos:\\

\textbf{Capítulo 1. Preliminares. }
\begin{itemize}
\item {Planteamiento del Problema; en él se plantean las razones en las cuales esta basado el proyecto de investigación.}
\item {Objetivos, se expone el objetivo general del proyecto de investigación y se identifican los objetivos específicos mediante los cuales se lleva a cabo el desarrollo del proyecto.}
\item {Justificación e importancia, se establecen los motivos por cuales el desarrollo del proyecto de investigación es relevante}
\item {Alcance y limitaciones, se detalla la funcionalidad del proyecto de investigación.}
\end{itemize}
\textbf{Capítulo 2. Fundamentos teóricos}
\begin{itemize}
\item {Antecedentes, se exponen revisiones de investigaciones previas que posean relación directa con el tema del actual proyecto de investigación.}
\item {Bases teóricas, se definen un conjunto de conceptos que proporcionan conocimiento acerca del objeto de estudio.}
\end{itemize}
\textbf{Capítulo 3.Fundamentos metodológicos}
\begin{itemize}
\item {Enfoque de la investigación, se plantea el enfoque utilizado para llevar a cabo la investigación.}
\item {Tipo o nivel de investigación, se expone el tipo de investigación y nivel de investigación, según el grado de profundidad que abarca y la naturaleza de los datos.}
\item {Diseño de la investigación, se establece la estrategia general que adopta el investigador para responder al problema planteado.}
\item {Metodología, se describen el conjunto de técnicas, herramientas y pautas utilizadas para llevar a cabo el proyecto de investigación.}
\end{itemize}
\chapter{Preliminares}

\section{Planteamiento y formulaci\'on del problema}

Una red neuronal artificial (ANN) es un modelo que crea una relacion entre una configuracion de senales de entrada y una senal de salida usando un modelo derivado de nuestro entendimiento de como el cerebro responde a determinados estimulos de entradas sensoriales. Asi como un cerebro usa ese conjunto de celulas interconectadas llamadas neuronas, asi las ANN usan una red de neuronal artificiales o nodos para solucionar problemas aprendidos (Lantz, 2015).\\

En general, las ANN son aprendices versatiles que pueden ser aplicarse a cais cualquier tarea de aprendizaje en cualquier contexto, clasificacion, prediccion, reconocimiento de patrones e incluso pudieramos hablar de tareas de supervision de procesos. Las ANN, se aplican mejor a los problemas cuyos datos de entrada y salida son bien definidos o bastante simples, sin embargo el proceso que relaciona la entrada con una determinada salida es en extremo complejo, por eso se conocen como un metodo de caja negra (Lantz, 2015).\\

De una manera muy arcaica, las ANN han sido usadas desde hace cincuenta anos para simular como el cerebro humano es capaz de resolver problemas. Primero surgieron las simples funciones logicas AND y OR, que ayudaron a los cientificos a entender como biologicamente el cerebro opera. Sin embargo, como el avance tecnologico y computacional ha sido incrementado y continua creciendo potencialmente durante los ultimos anos, las ANN ha incrementado su complejidad siendo usadas frecuentemente en multiples problemas como:

\begin{enumerate}
    \item{Programas de reconocimiento de voz y escritura, como los que usan los correo de voz
de los servicios de transcripción y máquinas clasificadoras de correo postal.}
	\item{La automatización de dispositivos inteligentes como los controles ambientales de un edificio de oficinas o automóviles autodirigidos y drones autoguiados.}
	\item{Modelos sofisticados de patrones climáticos y otros fenómenos científicos, sociales o económicos.}
	\item{La clasificacion y reconocimiento de cualquier cosa, basado en sus atributos y caracteristicas.}
\end{enumerate} 

Existen muchos metodos y lenguajes que permiten la estructuracion y desarrollo de ANN, destacandose lenguajes como Python y softwares como Statgraphics y SPSS, sin embargo como el objetivo de esta investigacion es trabajar con una variable de tipo espacial que amerita la aplicacion de regresion espacial antes para la normalizacion de los datos recogidos y observados, se propone el uso del lenguaje  R con el paquete neuralnet para el manejo y construcciond de la red neuronal.\\

R es un lenguaje de programación y entornos para graficos y calculos estadisticos, es una implementacion diferente al lenguaje S (Chambers, 1998), y se desarrolla bajo un proyecto de software bajo Licencia Publica General (General Public License, GNU) de la Free Software Foundation (https://www.fsf.org/) en forma de codigo fuente. Se compila y se ejecuta en una amplia variedad de plataformas entre ellas FreeBSD, Linux, Windows y MacOS. (The R Project for Statistical Computing, 2018).\\

El paquete neuralnet es un paquete del lenguaje R de muy sencilla utilizacion que permite la aplicacion, visualizacion e implementacion de redes neuronales. El paquete permite configuraciones flexibles a traves de
eleccion personalizada del error y función de activacion. Ademas, el calculo de pesos generalizados esta implementado (CRAN-R Project, 2016).\\

Esta propuesta permitira ajustar y entrenar una red neuronal artificial probabilistica que permita la clasificacion de densidades de siembra de papa yema de huevo (Solanum Phureja) teniendo como entrada las siguientes variables: 

\begin{enumerate}
    \item{\textbf{Peso total del cultivo.} Variable cuantitativa expresada en gramos (gr), que indica el peso de todos los tuberculos recogidos en la siembra.}
	\item{\textbf{Total de tuberculos recogidos.} Variable cuantitativa que indica el numero de tuberculos recogidos en la siembra.}
	\item{\textbf{Diametro ponderado de todos los tuberculos.} Variable cuantitativa expresada en centimetros (cm), que indica el calculo del diametro ponderado (por los tuberculos ser ovoides) de cada tuberculo recogido en la siembra.}
	\item{\textbf{Calibre de los tuberculos recogidos.} Variable cualitativa, con 4 valores posibles que definen a que calibre pertenece un tuberculo. Siendo los calibres dividivos en tuberculos menores a dos centimetros (<2cm), entre dos y cuatro centimetros (2-4cm), entre cuatro y seis centimetros (4-6cm) y mayores a seis centimetros (6cm).}
\end{enumerate}

La densidad de siembra es la distancia entre plantas y leras o sencillamente cuantas plantas son sembradas por metro cuadrado de siembra.\\

El objetivo es comparar los resultados de clasificacion alimentando la red con datos observados y con los mismos datos normalizados espacialmente a traves de regresion espacial, para observar si es necesario el analisis de los datos recogidos antes de la clasificacion y a su vez concluir hasta que punto la densidad de siembra afecta las variables antes mencionadas. Otra de las pruebas que se plantea usar en esta investigacion es realizar los analisis eliminando variables de entrada atisbando cuales variables permiten y generan una mejor clasificacion.\\

La curva ROC es una herramienta estadistica utilizada en analisis de clasificacion para determinar la capacidad discriminante de una prueba diagnostica dicotomica. Esta capacidad discriminante esta sujeta al valor umbral elegido de entre todos los posibles resultados de la variable de decision, es decir, la variable por cuyo resultado se clasifica (en nuestro caso la densidad de siembra) en un determinado grupo. La curva es el grafico resultante de representar, para cada valor umbral, las medidas de sensibilidad y especificidad de la prueba diagnostica (Benavides, SF).\\

Posterior a las diversas clasificaciones planteadas, se propone la construccion en lenguaje R de las curvas ROC correspondientes para determinar cuan optima es la clasificacion resultante, permitiendo asi llegar a las conclusiones deseadas. 

\section{Objetivos}

\subsection{Objetivo General}

Clasificar tuberculos de papa criolla para diferentes densidades de siembra empleando redes neuronales probabilisticas.

\subsection{Objetivos Espec\'ificos}
 
\begin{itemize}
\item{Disenar un algoritmo que permita realizar el estudio de regresion espacial a los datos de entrada de cultivo de pap criolla.}
\item{Implementar y entrenar una red neuronal artificial probabilistica que permita la clasificacion de tuberculos de papa para diferentes densidades de siembra.}
\item{Conseguir a traves de curvas ROC una clasificacion optima de  tuberculos de papa para diferentes densidades de siembra para diferentes casos de prueba.}
\end{itemize}

\section{Justificaci\'on e Importancia}

Colombia es el mayor productor, consumidor y exportador de papas diploides en el mundo; tiene una ventaja competitiva notable debido a ser centro de diversidad y poseer gran aceptacion por los consumidores debido a las caracteristicas del tuberculo. Adicionalmente, en este pais se ha desarrollado una amplia tradicion como cultivo tecnificado, con potencial de industrializacion y exportacion (Rodriguez \textit{et al}, 2009). Ademas de contar con uno de los recursos geneticos que ofrece las mayores oportunidades de exportacion como alimento procesado exclusivo y sin competencia. Desafortunadamente, muchas de las estrategias de manejo del cultivo de la papa, han sido adaptadas a papa criolla, creando un sistema productivo poco eficiente (Piñeros, 2009). Los altos precios de los insumos agricolas y el mal manejo de los cultivos agronomicamente provoca  baja productividad y amenaza la competitividad del sistema de produccion, por lo que es importante identificar factores limitantes del rendimiento y desarrollar practicas innovadoras para el cultivo, tales como una gestion nutricional integrada y equilibrada, que es una de las practicas mas eficientes para garantizar a la planta la oportunidad de expresar su potencial genetico que eventualmente se reflejara en una mejor calidad y rendimiento (Lopez \textit{et al}, 2014).\\
 
Varias investigaciones revelan que el tamano promedio de los tuberculos, su variabilidad y su conteo, definen una unica distribucion, presentando mayor variabilidad de rangos de tamano en la etapa final de llenado, con aumento del rendimiento de los mayores tamanos por cuenta de los tamanos inferiores, lo que sugiere que un tuberculo que pasa de un calibre inferior al siguiente, no es sustituido por otro de calibre anterior, ademas, parece existir una correlacion negativa entre la variabilidad relativa del tamano del tuberculo y el numero de tuberculos por unidad de area (Struik, 1991). Es importante reconocer, que debe tenerse cuidado con la interpretacion de esta correlacion, pues hasta cierto punto, la naturaleza de los datos asociados al numero de tuberculos por planta pudiera ser similar a la de los datos composicionales, es decir, si un componente aumenta, el otro esta forzado a disminuir, lo que genera correlaciones sin conexion logica en datos verdaderamente composicionales, sin embargo, el numero de tuberculos por calibre no suma una constante, pudiera ocurrir que su maximo posea una distribucion especifica, entonces una mayor cantidad de tuberculos de un calibre podria provocar un numero menor de tuberculos en otro calibre especifico, lo que en esencia puede generar correlaciones en las que debe tenerse mucho cuidado al momento de su interpretacion. La importancia del cultivo obliga a muchos investigadores a realizar diferentes estudios para mejorarlo segun el uso que vaya a darse a los tuberculos, generando nuevas variedades resistentes a plagas y enfermedades y de facil adaptacion a diferentes climas, procurando hacer uso del espacio de siembra de forma optima, por lo que varios estudios evidencian el estudio de la densidad de siembra para evaluar sobre todo el rendimiento. En 2017, Bernal evaluo en lugar del rendimiento, un indicador que se asocia directamente como lo es el calibre de los tuberculos, los cuales en la practica se manipulan en cuatro categorias de diametro promedio (hasta 2 cm, de 2 a 4 cm, de 4 a 6 cm y más de 6 cm). La naturaleza de esta variable imposibilita usualmente la comparacion de la respuesta (conteos de tubérculos) para cada densidad de siembra (definidas como 30cm*100cm, 40cm*100cm y 50cm*100cm) mediante analisis de varianza, pues en el caso de conteos, otras distribuciones como la Poisson y la Binomial negativa se adaptan mucho mejor al tipo de dato generado. Los modelos clasicos de regresion Poisson y binomial y negativa, en sus modalidades usuales o en la opción inflada por ceros (Cameron, 1998) en datos de conteo pertenecen a la familia de modelos lineales generalizados (Zeileis, 2008) y los desarrollos recientes permiten generar varias estadisticas que permiten su comparacion con sus contrapartes sin ceros en exceso, lo que resulta util en la eleccion del mejor modelo para relacionar predictores y respuesta en conteos.\\

Sin embargo, los estudios realizados hasta ahora son todos haciendo suposiciones del tipo lineal y haciendo modificaciones que permitan asi su estudio a traves de anova o regresion lineal tomando en cuenta los patrones de vecindad, mas la propuesta consiste en clasificar los datos a traves de una red neuronal que no amerita de suposiciones estadisticas o matematicas, que solo aprendiendo de lo observado es capaz de realizar una clasificacion de densidades de siembra, anadiendo aun asi para conseguir un mejor reusultado un analisis de regresion espacial previo, con el objetivo de solo con datos de siembra previos conocer a que densidad de siembra se debe cosechar para obtener un deseado calibre para fines comerciales e industriales. 

\section{Alcance y Limitaciones}

La red neuronal artificial probabilistica a implementar tiene como proposito la clasificacion de la densidad de siembra en papa criolla o yema de huevo, siendo util para incrementar las posibilidades de obtener un determinado y deseado calibre de tuberculos conociendo la densidad a la que se debe sembrar, anadiendo tambien para su optimizacion analisis de regresion espacial y un conjunto de pruebas que van desde la variacion de las variables de entrada hasta la utilizacion de los datos observados, para definir que tan buena o no es la clasificacion. Sin embargo las pruebas se relizaran bajo solo un conjunto de datos reales que fueron tomados bajo las medidas de cuatro calibres de tuberculos y tres densidades de siembra. \\

Los datos disponibles del cultivo de papa se tienen en base a tres densidades de siembra, cantidad de plantas sembradas, cantidad de tuberculos recogidos, peso total del cultivo, diametro ponderado de cada tuberculo y calibre de los tuberculos y forman parte de la investigacion de Maestria del estudiante de la Universidad Nacional de Colombia, Nelson Bernal.\\

Los resultados de las pruebas sobre los escenarios planteados seran validados estadisticamente con ayuda del analisis de las curvas ROC generados por estos mismos resultados.\\

\chapter{Fundamentos te\'oricos}

\section{Antecedentes}

La clasificacion por medio de redes neuronales ha sido un hito ya marcado en el campo agronomico, muchos estudios se han realizado con el objetivo de analizar ciertos comportamientos de plantas y los beneficios que se puedan sacar de ellas. En 2011, el grupo de investigacion de sistemas de procesamiento y control de senales de la Universidad Nacional Tenaga de Malasia desarrollo un sistema de inteligencia con un enfoque novedose para la clasificacion de frutas usando tecnicas de procesamiento de imagenes digitales y redes neuronales artificiales, con el objetivo de desarrollar un metodo de clasificacion rapido con una meta del 100\% de eficiencia. El estudio se realizo con cinco frutas, manzanas, platanos, zanahorias, mangos y naranjas, extrayendo de ellas siete caracteristicas en funcion de la forma y el color. La captura de las imagenes se realizo con una camara digital convencional y las manipulaciones a los datos y construccion de la red con el software MATLAB. Los resultados obtenidos durante esta investigacion fueron de gran avance en el campo de reconocimiento de patrones en imagenes.\\

En 2011, Mayabiro E presento en la Universidad Nacional Experimental del Tachira, un prototipo sobre el entorno MATLAB para el calculo de la tasa de germinacion de plantulas de pimenton previamente segmentadas. El entorno desarrollado permitia establecer una clasificacion de las plantulas, hojas u objetos de la misma por medio de redes neuronales. Se realizo el entrenamiento de multiples redes neuronales multicapas con algoritmos de retropropagacion, donde aunque variaban las capas intermedias de las redes y sus funciones de transferencia fueron entrenadas con los mismos  datos de entrada, validandolas con una base de datos de pruebas para seleccionar al final una con salidas similares a las deseadas.\\

Otro estudio realizado en 2013 por Stephen Gang Wu consistia en el estudio teorico de tecnicas de procesamiento de imagenes y datos para el reconocimiento automatico de hojas para la clasificacion de plantas. Doce caracteristicas de las plantas fueron extraidos y distribuidas en cinco variables principales que constituian el vector de entrada de una red neuronal artificial probalistica, que habis sido entrenada con 1800 hojas para clasificar 32 tipos de plantas con una precision superior al 90\%, el autor aseguraba que su metodologia de implementacion de la PNN era facil y rapida en comparacion de otras investigaciones similares.\\

En 2017, Bernal N realizo un estudio de campo con el cultivo de papa criolla para evaluar la influencia de la densidad de siembra asociada a distancias entre plantas de 30,40 y 50 cm y distancias entre surcos de 100 cm sobre el conteo de tuberculos de calibres inferiores a 2 cm, entre 2 y 4 cm, entre 4 y 6 cm, y de mas de 6 cm de diametro ponderado y sobre el peso fresco en gramos de los tuberculos. Los tuberculos cosechados se clasificaron y contaron mediante tamizado y se pesaron en su totalidad sin discriminar por calibre. Los modelos estadisticos empleados para modelar el comportamiento de la cosecha, evidenciaron el efecto significativo de la densidad de siembra sobre el conteo de tuberculos y calibre y se observo una razon aproximada de 40:40:20:1 desde el calibre menor al mayor. El efecto de la competicion, en todos los modelos probados resulto significativo, aumentando en la mayoria de los casos a medida que disminuia la distancia entre plantas, tanto en el patron de vecindad intrahileras como en el caso de inter e intrahileras.

\section{Bases Te\'oricas}

\subsection{Lenguaje R.}

R es un conjunto  integrado de \textit{software} de código abierto para el almacenamiento, manipulación, cálculo y visualización de datos para computación y gráficación estadística, puede ser compilado y ejecutado en  en Windows, Mac OS X y otras  plataformas UNIX (como Linux), se distribuye usualmente en formato binario (\url{https://www.r-project.org/about.html}, 2018). El proyecto de \emph{software} R fue iniciado por Robert Gentleman y Ross Ihaka. El lenguaje fue influenciado por  lenguaje S desarrollado originalmente en Bell Laboratories por John Chambers y sus colegas. Desde entonces ha evolucionado  para el cálculo estadístico asociado a diversas disciplinas para contextos académicos y comerciales. En R, la unidad fundamental de código compartible es el paquete, el cual agrupa código, datos, documentación y pruebas, y resulta simple de compartir con otros. Para enero del 2015 ya habían más de 6.000 paquetes disponibles en la Red Integral de Archivos de R, conocido comunmente por su acrónimo CRAN, el cual es el repositorio de paquetes . Esta gran variedad de paquetes es una de las razones por las cuales R es tan exitoso, pues es probable que algún investigador o académico ya haya resuelto un problema en su propio campo usando esta herramienta, por lo que otros usuarios simplemente podrán recurrir a ella para su uso directo o para llamarla en un nuevo código (Wickham,2015). \\


\subsection{RStudio.}

RStudio es un ambiente de desarrollo integrado (\textit{Integrated Development Environment}, IDE) que ofrece herramientas de desarrollo vía consola, editor de sintaxis que apoya la ejecución de código, así como herramientas para el trazado, la depuración y la gestión del espacio de trabajo.  RStudio está disponible para Windows, Mac y Linux o para navegadores conectados a RStudio Server o RStudio Server Pro (Debian / Ubuntu, RedHat / CentOS, y SUSE Linux) (\url{https://www.rstudio.com/about/}, 2018)
 

\subsection{Glosario}

\paragraph{R}: 



\chapter{Fundamentos Metodol\'ogicos}

	A continuaci\'on se plantea la estructura a seguir por el presente trabajo, detallando el enfoque, tipo, nivel y dise\~no de la investigaci\'on y la metodolog\'ia a implementar entre otros.
	
\section{Enfoque de la investigaci\'on}
	
	La presente investigaci\'on se desarrollar\'a siguiendo un enfoque cuantitativo, puesto que, como lo indican Pallela y  Martins (2012) , “la investigaci\'on cuantitativa requiere el uso de instrumentos de medici\'on y comparaci\'on, que proporcionan datos cuyo estudio necesita la aplicaci\'on de m\'odelos matem\'aticos y estad\'isticos, el conocimiento est\'a basado en hechos”.  Los datos a usar en el desarrollo de este trabajo investigativo fueron recolectados directamente de un cultivo y forman parte de la investigacion de Maestria del estudiante de la Universidad Nacional de Colombia, Nelson Bernal.\\
	
\section{Tipo o nivel de investigaci\'on}
	
	Este proyecto plantea un tipo de investigaci\'on de campo, seg\'un como lo indican Pallela y  Martins (2012),  la investigaci\'on de campo “consiste en la recolecci\'on directamente de la realidad donde ocurren los hechos, sin manipular o controlar variables” ya que permite indagar los efectos de la interrelaci\'on entre los diferentes tipos de variable en lugar de los hechos.\\

	En este punto se debe determinarla profundidad que abarca esta investigaci\'on, teniendo en cuenta que de acuerdo con  el nivel de la investigaci\'on es definido como “grado de profundidad con que se aborda un fen\'omeno u objeto de estudio” (Arias, 2012).\\

	En este sentido, se tiene que dadas las caracter\'isticas del proyecto, se asocia con un nivel descriptivo, tal como lo indican Pallela y  Martins (2012),  “hace \'enfasis sobre conclusiones dominantes o sobre como una persona, grupo o cosa se conduce o funciona en el presente” esto debido a que se medir\'an los datos extra\'idos sin alterarlos para ser mostrados en el sistema.\\

	Cuando se habla de un nivel descriptivo junto con una investigaci\'on de tipo de campo, en ella no se formulan hip\'otesis y las variables se enuncian en los objetivos de la investigaci\'on que se desarrollar\'a.
	
\section{Dise\~no de la investigaci\'on}
	
Seg\'un Arias(2012), el dise\~no de la investigaci\'on es “la estrategia general que adopta el investigador para responder al problema planteado” (p.21) por lo que es vital establecer una correcta secuencia de pasos para elaborar el prototipo de software que dar\'a soluci\'on a la problem\'atica principal de la investigaci\'on.\\

Con este enfoque, se tiene que este trabajo seguir\'a un dise\~no no experimental, enfocado en el uso de informaci\'on existente, de acuerdo con lo dicho por Pallela y  Martins (2012) al definir el dise\~no no experimental como:\\

Es el que se realiza sin manipular en forma deliberada ninguna variable. El investigador no sustituye intencionalmente las variables independientes. Se observan los hechos tal y como se presentan en su contexto real y en un tiempo determinado o no, para luego analizarlos. Por lo tanto, este dise\~no no se construye una situaci\'on espec\'ifica sino que se observan las que existen. Las variables independientes ya han ocurrido y no pueden ser manipuladas, lo que impide influir sobre ellas para modificarlas. (p.81)\\

Esto indica que no hay manipulaci\'on de variables. Esta investigaci\'on presenta una modalidad de proyecto especial que, como lo indican Pallela y  Martins (2012), los proyectos especiales “destinados a la creaci\'on de productos que puedan solucionar deficiencias evidenciadas, se caracterizan por su valor innovador y aporte significativo” (p.92), ya que se crear\'a un \emph{software} aplicable al \'area de estudio.\\

\section{Metodolog\'ia}

Los pasos a seguir en el desarrollo de esta investigacion seran descritos a continuacion siendo basados en los antecedentes y estudios realizados y las pautas est\'andar establecidas para la creación de paquetes y extensiones en R.\\

\noindent
\textbf{Creación del esqueleto del paquete.}\\
En esta etapa se dise\~naran y crearan los directorios, ficheros y objetos que conformaran el paquete.\\

\noindent
\textbf{Registrar el m\'etodo para el env\'io y uso de funciones.}\\

Es esta etapa del desarrollo se estableceran las dependencias sobre los paquetes de la base fuente de código R y sus métodos de conexión, considerando el manejo de versiones y los criterios de mantenimiento, además  se estableceran  los espacios de nombre o las estrategias para la búsqueda y utilización de las variables; unificando estos criterios a las funciones que seran dise\~nadas.\\

\noindent
\textbf{Dise\~no y codificaci\'on de las funciones.}\\

Los m\'etodos para el dise\~no de las funciones primarias en R ser\'an los diagramas de flujo; y para su codificaci\'on se seguir\'an las normas de estilo para codificaci\'on en R, sugeridas por Wickham (2015) y por el creador del paquete \emph{formatR} Xie(2017), adem\'as se establecer\'an la dependencia con las funciones de c\'odigo base y las recomendadas para desarrollo en R.\\

\noindent
\textbf{Explorar y manipular la data.}\\

Los datos tomados de la investigacion de Bernal, son datos reales observados de un cultivo de tuberculos de papa criolla que no han sido tratados estadisticamente o modelados, es por eso que en esta etapa se disenara un algoritmo para la carga de los mismos en estructuras de datos en R, para luego ser modelados y analizados espacialmente por medio de regresion espacial y probando de igual forma su normalizacion bivariante.\\

\noindent
\textbf{Disenar y entrenar la red neuronal probabilistica.}\\

En esta estapa se definiran las variables de entrada y se declararan las neuronas patrones y las clases de salida que permitiran la clasificacion, para proceder con el entrenamiento de la red que se hara a traves del metodo de jackknifing para determinar el parametro de escalamiento correcto que permita la mejor clasificacion.\\

\noindent
\textbf{Pruebas unitarias de las funciones.}\\

Debido a que los paquetes en R est\'an conformados, entre otros elementos por las funciones primarias, a cada una de ellas se les realizaran pruebas unitarias en dos fases, la primera con datos sint\'eticos que permitan comprobar cada estado del diagrama de flujo, que esquematiza la soluci\'on num\'erica que permite el c\'alculo de cada uno de los \'indices fisiol\'ogicos para entre otra herramientas pueden utilizarse los paquetes de R\emph{RUnit} (Zenka, 2015) y \emph{testthat} (Wickham, 2017), y la segunda etapa donde cada funci\'on asociada a un \'indice fisiol\'ogico se le realizan prueban con los conjuntos de datos de prueba que formaran parte integral del paquete y con los cuales se desarrollaran los ejemplos pr\'acticos que conformaran la documentaci\'on que acompa\~a al paquete R.\\

\noindent
\textbf{Implementar y realizar pruebas de clasificacion.}\\

El objetivo de esta etapa es definir los casos de prueba que seran usados para buscar la mejor clasificacion de densidad de siembra de tuberculos de papa criolla y realizar las pruebas con los diferentes escenarios a preparar.\\

\noindent
\textbf{Comparacion de resultados.}\\

En este paso se procedera a la construccion de las curvas ROC asociadas a los casos de prueba planteados para asi obtener las mejores clasificaciones, permitiendo concluir las mejores relaciones entre variables, el mejor manejo de los datos y si los mismos deben ser modelados o es suficientemente buena una clasificacion con datos observados a pesar de la dependencia espacial existente.\\

\noindent
\textbf{Graficar y documentar resultados.}\\

Finalmente se procedera a la graficacion por medio de R de los resultados, que sustenten las conclusiones realizadas y a relizar la documentacion que permita el uso de esta investigacion en futuras aplicaciones de campo.\\  
  
\noindent
\textbf{Chequear la carga del paquete.}\\

En esta etapa del desarrollo se utilizaran las funciones de chequear paquete que ofrece el código R; cuya finalidad es verificar cada fichero del árbol de carpetas asociadas a cada elemento de la estructura o esqueleto del paquete, que a su vez creara el archivo de documentación en LaTeX y/o HTML, compilará el código fuente y creará las librerías de enlace dinámico (\emph{dynamic link library} DLL).\\  

\noindent
\textbf{Construcci\'on del m\'etodo de distribuci\'on del paquete.}\\

Se seleccionara la forma de distribución del paquete desde el repositorio local, creando los ficheros fuentes  (en formato  \emph{tarball})  y en binario.\\

\section{Aspectos administrativos}

\vspace{1 cm}
La realización de la investigación será planificada según lo establecido en el siguente diagrama:\\

\begin{figure}[!ht]
\begin{center}

\begin{ganttchart}[y unit title=0.4cm,
y unit chart=0.5cm,
vgrid,hgrid,
title height=1,
bar/.style={draw,fill=cyan},
bar incomplete/.append style={fill=yellow!50},
bar height=0.7]{1}{16}
 \gantttitle{Semanas}{16}\\
 \gantttitle{Julio}{2}
 \gantttitle{Agosto}{4} 
 \gantttitle{Septiembre}{4}
 \gantttitle{Octubre}{5}\\
 \gantttitle{Noviembre}{1}\\
 \gantttitlelist{1,...,2}{1}
 \gantttitlelist{1,...,4}{1}
 \gantttitlelist{1,...,4}{1} 
 \gantttitlelist{1,...,5}{1} 
 \gantttitlelist{1,...,1}{1}\\
 \ganttbar{\tiny{Determinación de los lineamientos para el dise\~no del paquete}}{1}{1} \\
 \ganttbar{\tiny{Creación del esqueleto del paquete}}{2}{2} \\
 \ganttbar{\tiny{Dise\~no de los algoritmos para la biblioteca de funciones}}{2}{3} \\
 \ganttbar{\tiny{Codificación de la biblioteca de funciones}}{3}{9} \\
 \ganttbar{\tiny{Explorar y manipular la data}}{3}{4} \\
 \ganttbar{\tiny{Disenar y entrenar la red neuronal probabilistica}}{5}{8} \\
 \ganttbar{\tiny{Implementar y realizar pruebas de clasificacion.}}{8}{12} \\
 \ganttbar{\tiny{Comparacion de resultados}}{12}{14} \\
 \ganttbar{\tiny{Graficar y documentar resultados}}{15}{16} \\
 \ganttbar{\tiny{Prueba de la biblioteca de funciones}}{9}{12} \\
 \ganttbar{\tiny{Documentación del paquete}}{4}{12} \\
 \ganttbar{\tiny{Chequeo del paquete}}{10}{14} \\
 \ganttbar{\tiny{Creación del método de distribución del paquete}}{14}{15} \\
 \ganttbar{\tiny{Pruebas de distribución del paquete}}{15}{16} \\
 \ganttbar{\tiny{Realización del informe del proyecto especial de grado}}{2}{16} \\
\end{ganttchart}

\end{center}
\caption{Diagrama de Gantt con la planficación del proyecto especial de grado}
\end{figure}
%% Los cap'itulos inician con \chapter{T'itulo}, estos aparecen numerados y
%% se incluyen en el 'indice general.
%%
%% Recuerda que aqu'i ya puedes escribir acentos como: 'a, 'e, 'i, etc.
%% La letra n con tilde es: 'n.

\chapter*{Referencias Bibliográficas}

\noindent
Available CRAN Packages By Name. Disponible en:\url{ https://cran.r-project.org/web/packages/available_packages_by_name.html}. Consultada Junio 2018.\\

\noindent
Lanz Brett, Reino Unido. Machine Learning with R. 2015. \\

\noindent
Bohorquez Ingrid, Ceballos Ermilson. Algunos conceptos de la econometría espacial y el análisis exploratorio de datos espaciales. 2008.

\noindent
Stephen Gang Wu, Forrest Sheng Bao, Eric You Xu, Yu-Xuan Wang, Yi-Fan Chang, Qiao-Liang Xiang. A Leaf Recognition Algorithm for Plant Classification Using Probabilistic Neural Network. 2007.

\noindent
Nur Badariah Ahmad Mustafa, Kumutha Arumugam, Syed Khaleel Ahmed, Zainul Abidin Md Sharrif. Classification of Fruits using Probabilistic Neural Networks - Improvement using Color Features. 2011.

Bernal Nelson. Modelado del Calibre y la competición intra-específica por rendimiento de tubérculos de papa variedad Solanum phureja bajo diferentes densidades de siembra. 2017.




%% Incluir la bibliograf'ia. Mirar el archivo "biblio.bib" para m'as detales
%% y un ejemplo.


\end{document}
